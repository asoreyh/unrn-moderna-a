\documentclass[a4paper,12pt]{article}
\usepackage[spanish]{babel}
\hyphenation{co-rres-pon-dien-te}
%\usepackage[latin1]{inputenc}
\usepackage[utf8]{inputenc}
\usepackage[T1]{fontenc}
\usepackage{graphicx}
\usepackage[pdftex,colorlinks=true, pdfstartview=FitH, linkcolor=blue,
citecolor=blue, urlcolor=blue, pdfpagemode=UseOutlines, pdfauthor={H. Asorey},
pdftitle={Física Moderna A - Asorey - 2017}]{hyperref}
\usepackage[adobe-utopia]{mathdesign}

\hoffset -1.23cm
\textwidth 16.5cm
\voffset -2.0cm
\textheight 26.0cm

%----------------------------------------------------------------
\begin{document}
\title{
{\normalsize{Universidad Nacional de Río Negro - Profesorados de Física}}\\
Física Moderna A \\ Postulados de la Mecánica Cuántica\\}
\author{Asorey}
\date{2017}
\maketitle

\begin{enumerate}
\setcounter{enumi}{22}      %% Offset en numero de problema


\item {\bf{$\Delta p$}}:\\
	Utilice el principio de incertidumbre para analizar la validez de la siguiente
		afirmación: ``Un nuevo método de medición permite determinar la
		posición de un electrón con una resolución espacial de $0.05$\,nm y
		medir, en forma simultánea, la velocidad del mismo con una precisión de
		$10$\,m/s''. Justifique.

\item{\bf{$\Delta E$}}:\\
	Se dice que un estado excitado de un átomo es metaestable cuando el
		electrón puede permanecer en ese estado por tiempos largos comparados
		con los tiempos típicos de una transición atómica (ms para un estado
		metaestable frente a ns para un estado normal). Imagine un átomo con un
		estado normal de una duración de $1$\,ns y un estado metaestable de
		$1$\,ms de duración. ¿Cuál es la incertidumbre que puede obtenerse para
		la medición de la energía de esos dos estados? Justifique.		

\item {\bf{Mesón eta}}:\\
	El mesón $\eta$ es una partícula inestable con una masa $m=549$\,MeV/c$^2$
		y una vida media de $7\times 10^{-19}$\,s. ¿Cuál es la incertidumbre
		para la medición de la masa de la partícula?

\item{\bf{Microscopio electrónico, 1}}:\\
	Para un microscopio electrónico, calcule cual debe ser el voltaje de
		aceleración necesario para que la longitud de onda de los electrones
		acelerados sea de $0.04$\,nm suponiendo que inicialmente el electrón
		está en reposo. ¿Qué pasaría con el voltaje si se usaran protones?

\item{\bf{Microscopio electrónico, 2}}:\\
	Calcule la energía de un fotón con una longitud de onda de $1$\,nm, y
		compárela con la energía cinética de un electrón de la misma longitud
		de onda.

\item{\bf{Microscopio electrónico, 3}}:\\
	Calcule la energía cinética de un electrón cuya longitud de onda es igual a
		la de un fotón de $0.1$\,MeV.

\item {\bf{Velocidad de grupo y velocidad de fase}}:\\
	A partir de las relaciones de De Broglie para una partícula de masa $m$ que
		se mueve a velocidad $v$, $E=hf$ y $p=h/\lambda$, calcule la velocidad
		de grupo $v_g=\omega / k$ y la velocidad de fase $v_p=\partial
		\omega/\partial k$ como función de la velocidad de la partícula $v$.

\item{\bf{Función de onda}}:\\
	Imagine que se propone una función de onda para una partícula confinada en
		una región unidimensional para $x\geq0$ de la forma $\psi(x)=A
		e^{\lambda x}$, con $A,\lambda \epsilon \mathbb{R}$. Diga para que
		valores de $\lambda$ $\psi$ puede ser una buena función de onda, y
		luego obtenga una función para $A$. Indique las unidades de $\lambda$ y
		$A$. 

\item{\bf{Función de onda}}:\\
	Diga si las siguientes funciones pueden ser funciones de onda. Si lo son,
		proponga valores para normalizar las mismas, y si no lo son explique
		porque y proponga como pueden serlo:
		\begin{enumerate}
			\item $\psi(x) = A |x|, \qquad -2 < x < 2$;
			\item $\psi(x) = A, \qquad  A\epsilon \mathbb C, x \geq 0$;
			\item $\psi(x) = A \tan x, \qquad x \geq 0$;
			\item $\psi(x) = A e^{a x}, \qquad a>0, x\geq 0$;
			\item $\psi(x) = A e^{a x}, \qquad a>0, x\leq 0$;
			\item $\psi(x) = A e^{a x^2}, \qquad a<0, -\infty < x < \infty$;
			\item $\psi(x) = A \sqrt{a x}, \qquad a>0, 0 < x < 3$;
		\end{enumerate}

\item{\bf{Probabilidades}}:\\
	Un sistema está descripto por la siguiente función de onda: $\psi(x)=A
		e^{-(x/\alpha)^2}$, con $A$ y $\alpha$ constantes positivas. Si
		$\alpha=a_0$ (el radio de Bohr), obtenga el valor de $A$ que normalice
		la función y luego calcule cuál es la probabilidad de encontrar a la
		partícula en la región $-a_0/2 \leq x \leq a_0/2$. Luego, diga cual
		es la probabilidad de encontrar a la partícula en la región $100 a_0
		\leq x \leq 101 a_0$.

\item {\bf{Fase temporal}}:\\
	La ecuación de Schrödinger, 
	$$\left (-\frac{\hbar^2}{2m}\frac{\partial^2}{\partial x^2} + \hat
	U(x,t)\right ) \Psi(x,t) = i \hbar \frac{\partial}{\partial t} \Psi(x,t)$$
	se reduce a la llamada ecuación de Schrödinger independiente del tiempo o
	bien, ecuación de Schrödinger estacionaria, si el potencial no depende
	explícitamente del tiempo, es decir, $U(x,t)=U(x)$: 
	$$\left (-\frac{\hbar^2}{2m}\frac{\partial^2}{\partial x^2} + \hat
	U(x)\right ) \Psi(x,t) = E \Psi(x,t).$$
	Demuestre que si $\psi(x)$ es una solución de la ecuación estacionaria con
	energía $E$, entonces $\Psi(x,t)=\psi(x)\varphi(t)$, donde
	$\varphi(t)=e^{-i \omega t}$ es una solución de la ecuación de Schrödinger
	bajo una cierta relación entre $\omega$ y $E$. Encuentre dicha relación. 

\item{\bf{Coseno cuadrado}}:\\
	La función de una onda de una partícula es $\psi(x)=A \cos^2(x), \quad
	-\pi/2 < x < \pi/2$. Calcule el valor de $A$ y luego obtenga la
	probabilidad de encontrar a la partícula en el intervalo $[0,\pi/4]$.

\item{\bf{Superposición}}:\\
	Demuestre que si $\Psi_1$ y $\Psi_2$ son funciones de onda cuánticas, y por
		ende, son soluciones de la ecuación de Schrödinger, entonces la
		combinación lineal de ambas, $\Psi=a_1 \Psi_1 + a_2 \Psi_2$ también lo
		es. 

\item {\bf{Paquete de onda}}:\\
	Resuelva el problema 39.68 de Física Universitaria Tomo 2 Ed. 12, pg. 1374.

\item {\bf{Electrón en una caja}}:\\
	Imagine un electrón confinado en un potencial unidimensional con la
	siguiente forma: $V(x)=0 \mathrm{si\ }0 < x < a_0$, con $a_0$ el radio de
	Bohr, e infinito en el resto del espacio. Encuentre los valores de cantidad
	de movimiento, energía y las funciones de onda para $n=1,2,3,4$. Luego
	calcule en que lugares será más probable encontrar al electrón, y en que
	lugares nunca lo encontraremos. Justifique. Finalmente, calcule la
	incertidumbre esperada para una medición de la cantidad de movimiento del
	electrón si suponemos $\Delta x=a_0$. Compare este valor con el valor de
	$p_1$.

\item {\bf{Auto en una cochera}}:\\
	Repita el ejercicio anterior, pero suponga ahora que tenemos un auto de
	masa $m=1000$\,kg encerrado en una cochera de 8 metros de longitud.
	Justifique los resultados usando el principio de correspondencia.

\item {\bf{Deuterio}}:\\
	El núcleo del átomo de deuterio, $^2$H, tiene un electrón y un protón y un
	diámetro aproximado de $2$\,fm. Calcule los niveles de energía del neutrón
	confinado en el núcleo y las funciones de onda de los primeros tres niveles
	($n=1,2,3$). Suponga para ello que se comporta como una caja de paredes
	infinitas de longitud igual al diámetro nuclear.  

\item {\bf{Ortonormalidad}}:\\
	Una importante propiedad de los autoestados de un sistema $\psi_n$ es que
	son ortonormales, es decir, que cumplen con la siguiente propiedad: 
	$$ \int_{-\infty}^{\infty} \psi_i^* \psi_j dV = \delta_{ij},$$
	donde $\delta_{ij}$ es la llamada delta de Kronecker, $\delta_{ij}=1$ si
	$i=j$ y $\delta_{ij}=0$ si $i\neq j$. Verifique esta propiedad para las
	soluciones de la partícula en una caja de longitud $L$, $$\psi_n(x) =
	\sqrt{\left( \frac{2}{L} \right)} \sin \left (\frac{\pi}{L} n x \right ).$$
\end{enumerate}
\end{document}
%%%%
