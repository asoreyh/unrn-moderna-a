\documentclass[a4paper,12pt]{article}
\usepackage[spanish]{babel}
\hyphenation{co-rres-pon-dien-te}
%\usepackage[latin1]{inputenc}
\usepackage[utf8]{inputenc}
\usepackage[T1]{fontenc}
\usepackage{graphicx}
\usepackage[pdftex,colorlinks=true, pdfstartview=FitH, linkcolor=blue,
citecolor=blue, urlcolor=blue, pdfpagemode=UseOutlines, pdfauthor={H. Asorey},
pdftitle={Física Moderna A - Asorey - 2017}]{hyperref}
\usepackage[adobe-utopia]{mathdesign}

\hoffset -1.23cm
\textwidth 16.5cm
\voffset -2.0cm
\textheight 26.0cm

%----------------------------------------------------------------
\begin{document}
\title{
{\normalsize{Universidad Nacional de Río Negro - Profesorados de Física}}\\
Física Moderna A \\ Los Inicios de la Mecánica Cuántica\\}
\author{Asorey}
\date{2017}
\maketitle

\section{Problemas}

Entregue al menos cinco de los siguientes ejercicios. 

\begin{enumerate}
\setcounter{enumi}{0}      %% Offset en numero de problema

\item {Átomo de Bohr}

En base al modelo atómico de Bohr:

\begin{enumerate}
\item Determine el radio de las órbitas permitidas. Calcule el radio de la
primer órbita de Bohr para el átomo de
hidrógeno.
\item Muestre que la energía del electrón está cuantizada (admite sólo valores
discretos). Calcule la energía correspondiente a un electrón en la primera
órbita de Bohr en un átomo de hidrógeno (estado fundamental del átomo). Dibuje
el diagrama de niveles de energía para un átomo de hidrógeno.
\item Justifique la utilización de mecánica clásica en lugar de mecánica
relativista para átomos livianos (verificar que $v \ll c$).
\end{enumerate}

\item {Serie de Balmer}
De acuerdo al modelo atómico de Bohr, si un electrón se mueve en una de las
órbitas permitidas, su energía se mantiene constante (estado estacionario). El
electrón puede sufrir una transición ``no clásica'' de un estado estacionario a
otro de energía inferior emitiendo radiación electromagnética de frecuencia
$\nu = \Delta E/h$, siendo $\Delta E$ la diferencia de energía entre los dos
estados involucrados y $h$ es la constante de Planck.

\begin{enumerate}
\item Balmer encontró una fórmula empírica para representar las longitudes de
onda de las líneas correspondientes al espectro de emisión del hidrógeno que se
encuentra en la región visible (esta serie de líneas espectrales se conoce como
serie de Balmer): $\lambda_n = a n^2 / (n^2 - 4)$. Determine el valor de la
constante $a$ (La serie de Balmer corresponde a las siguientes líneas
espectrales: $\lambda_n=656.3, 486.1, 434.1, 410.2, 397.0, 388.9, 383.5,
346.6$\,nm).
\item Determine en qué región del espectro electromagnético se encuentran las
siguientes series del hidrógeno:
\begin{enumerate}
\item Serie de Lyman ($n_f = 1$)
\item Serie de Paschen ($n_f = 3$)
\end{enumerate}
$n_f$ representa el número cuántico correspondiente al estado hacia el cual el
electrón experimenta la transición.
\end{enumerate}

\item {Temperatura}

¿A qué temperatura la energía cinética molecular promedio del hidrógeno gaseoso
será igual a la energía de ionización del átomo de hidrógeno? (ayuda: recuerde
la expresión $E=k T$, donde $k=1.3806 \times 10^{-23}$\,J\,K$^{-1}$ es la
constante de Boltzmann). ¿Cuál debería ser la longitud de onda de un fotón para
ionizar a un átomo de hidrógeno?

\item {Emulando a Bohr}

Considere un hipotético átomo con un electrón, cuyos niveles de energía no son
los del átomo de hidrógeno, pero que obedece el postulado de Bohr, es decir, el
electrón no irradia energía mientras permanece en uno de sus estados orbitales,
y la radiación ocurre solamente cuando el electrón pasa desde un estado de
mayor energía a otro de menor energía, emitiendo un cuanto de radiación de
energía $E=h\nu$ igual a la diferencia de las energías de los estados. 

Las longitudes de onda de las primeras cuatro líneas de la serie espectral que
terminan en $n = 1$ son: $120.0$\,nm, $100.0$\,nm, $90.0$\,nm y $84.0$\,nm. El
límite de longitudes de onda corta de esta serie es $80.0$\,nm.

\begin{enumerate}
\item Encuentre los valores de los primeros cinco niveles de energía de este
átomo en eV y dibuje el diagrama de niveles.
\item ¿Cuál es la energía de ionización?
\item ¿Cuál es la mínima energía que debe ser entregada al electrón en el
estado fundamental para poder observar la radiación correspondiente a la
transición de $n = 3$ a $n = 2$?
\end{enumerate}

\item {Efecto Auger}

En términos del modelo de Bohr, un átomo multi-electrónico se puede considerar
como un conjunto de electrones independientes ocupando distintas órbitas
hidrogenoides de Bohr. Considere un átomo del isótopo Be ($Z=4$) y suponga que
la órbita de menor energía puede alojar como máximo 2 electrones. Así, el
estado fundamental de este sistema contendrá dos electrones en el nivel $n = 1$
y dos en el nivel $n = 2$. Suponga ahora que uno de los electrones del nivel $n
= 1$ es removido por una colisión con un electrón externo, dejando al sistema
en un estado excitado. Basándose en el modelo de Bohr:

\begin{enumerate}
\item Calcule la longitud de onda del fotón emitido cuando un electrón del
nivel $n = 2$ migra a la vacancia del nivel $n = 1$.
\item Muestre que un segundo tipo de proceso es posible, en el cual uno de los
electrones del nivel $n=2$ es emitido espontáneamente (autoionización), en
tanto que el electrón restante del nivel $n=2$ se reacomoda en la órbita
inferior $n=1$. Esto se conoce como {\emph{efecto Auger}}. Ayuda: piense en la
conservación de la energía. ¿Cuáles serían los procesos que terminan generando
esta nueva configuración?
\item Calcule la energía cinética del electrón emitido en una transición Auger.
\end{enumerate}

\item {De Broglie}

Realice un gráfico de la longitud de onda de De Broglie en función de la
energía cinética de la partícula para el electrón ($m_e c^2 = 511$\,keV) y el
neutrón ($m_n c^2 = 940$\,MeV). En la misma figura incluya la longitud de onda
del fotón ($m_\gamma c^2 = 0$) en función de su energía. Analice el
comportamiento de la longitud de onda de De Broglie asociada a las partículas
para valores de energía cinética mucho mayores que la masa en reposo. Determine
para cada partícula el orden de magnitud de su energía cinética como para poder
observar difracción en un cristal, sabiendo que se requiere que la longitud de
onda asociada sea del orden de la separación entre los átomos del cristal,
$\sim 0.3$\,nm.

\item {Longitud de onda y energía de una partícula}

Una partícula de masa $m$ y carga $e$ es acelerada a través de una diferencia
de potencial $V$.  Encontrar la longitud de onda de la partícula. 

\item {Correspondencia}

Una bala de $40$\,g viaja a $1000$\,m\,s$^{-1}$. 

\begin{enumerate}
\item ¿Qué longitud de onda se le puede asociar? 
\item ¿Por qué no se revela la naturaleza ondulatoria de la bala por medio de
efectos de difracción? 
\item Si la incertidumbre en la medición de la velocidad de la bala es de
$0.01\%$, ¿cuál será la mínima precisión con que se podrá determinar su
posición si se la mide simultáneamente con la velocidad?
\end{enumerate}

\item {Incertidumbre}

Un átomo excitado puede irradiar en cualquier instante entre $t = 0$ y $t =
\infty$. No obstante, se observa que los átomos excitados decaen a estados de
menor energía en un tiempo promedio finito, el cual se conoce como tiempo de
vida media ($\tau$).El espectro de emisión del mercurio presenta una línea
intensa a la longitud de onda $\lambda =2536$\,\AA. Sabiendo que la vida media
del estado excitado correspondiente es de aproximadamente $10^{-8}$\,s, estimar
la indeterminación en la energía de este nivel.
\end{enumerate}
\end{document}
%%%%
