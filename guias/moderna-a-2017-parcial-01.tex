\documentclass[a4paper,12pt]{article}
\usepackage[spanish]{babel}
\hyphenation{co-rres-pon-dien-te}
%\usepackage[latin1]{inputenc}
\usepackage[utf8]{inputenc}
\usepackage[T1]{fontenc}
\usepackage{graphicx}
\usepackage[pdftex,colorlinks=true, pdfstartview=FitH, linkcolor=blue,
citecolor=blue, urlcolor=blue, pdfpagemode=UseOutlines, pdfauthor={H. Asorey},
pdftitle={Física Moderna A - Asorey - 2017}]{hyperref}
\usepackage[adobe-utopia]{mathdesign}

\hoffset -1.23cm
\textwidth 16.5cm
\voffset -2.0cm
\textheight 26.0cm

%----------------------------------------------------------------
\begin{document}
\title{
{\normalsize{Universidad Nacional de Río Negro - Profesorados de Física}}\\
Física Moderna A \\ Primer Parcial\\}
\author{Asorey}
\date{2017}
\maketitle

\begin{enumerate}
\setcounter{enumi}{0}

\item {\bf{Retina}}
	El desprendimiento de retina es una problema muy serio que puede desembocar
		en la ceguera en el caso de no tratarse. Uno de los tratamientos
		existentes utiliza dos láseres diferentes, que emiten fotones de
		$452$\,nm y $622$\,nm respectivamente. La emisión de cada láser es
		pulsada, con una duración de pulsos de $20$\,ms. Se sabe que para
		lograr soldar la retina, durante cada impulso, la potencia media es de
		$0.7$\,W en ambos casos. Responda para cada láser lo siguiente:
	\begin{enumerate}
		\item ¿Cuánta energía, medida en joules y en electronvolts, hay en
			cada pulso?
		\item ¿Cuál es la energía de cada fotón (en J y en eV)?
		\item ¿Cuántos fotones hay en cada pulso?
		\item Si se ilumina con este pulso un material cuya función de trabajo
			es $\varphi=1.2$\,eV, y suponiendo que el proceso es 100\%
			eficiente, ¿cuál es la energía máxima de los fotoelectrones
			resultantes? ¿Cuál sería la corriente eléctrica obtenida por un
			pulso?
	\end{enumerate}

\item {\bf{Muón}}
	El muón ($\mu^{-}$)es una partícula elemental que posee una carga negativa
		igual a la del electrón, pero su masa es $m_\mu = 105.7$\,MeV/c$^2$, es
		decir, aproximadamente 200 veces la masa del electrón. Imagine entonces
		que dos fotones incidentes, ambos con una longitud de onda $0.01$\,pm,
		son dispersados por sendos muones, inicialmente libres y en reposo, con
		un ángulo de $\theta=\pi/2$ y
		$\theta = \pi$. Conteste:
	\begin{enumerate}
		\item ¿Cuál es la cantidad de movimiento y la energía, en eV/c y eV
			respectivamente, del fotón incidente?
		\item ¿Cuál es la longitud de onda Compton del muón? Luego compárela
			con la del electrón.Justifique.
		\item Para cada caso de los fotones dispersados, diga cuál es longitud
			de onda, la energía y la cantidad de movimiento del fotón
			dispersado, y cuál es la cantidad de movimiento, la energía
			cinética y la velocidad del muón luego de la dispersión.
	\end{enumerate}

\item {\bf{Emisión}}
	Un cuerpo negro con un área total de $10$\,m$^2$ se encuentra a una
		temperatura $T=9000$\,K. La radiación proveniente del cuerpo negro
		incide sobre un volumen de Hidrógeno gaseoso.
		\begin{enumerate}
			\item ¿Qué tipo de espectro espera ver antes de que la radiación
				alcance el gas?
			\item ¿Cómo afectará a ese espectro la interacción con el volumen
				de gas?
			\item Haga un diagrama esquemático de la transición de Balmer
				correspondiente a la emisión de un fotón con la mayor longitud
				de onda posible, indicando claramente los valores de los
				números cuánticos y los valores de energía correspondientes al
				nivel inicial y final.
			\item Calcule la cantidad de transiciones por segundo que se
				observarán en el gas para cada una de las primeras cuatro
				líneas de Balmer.  Para ello obtenga la potencia espectral del
				cuerpo negro para cada una de esas líneas y obtenga el número
				de fotones que serán absorbidos por el gas en cada caso.
		\end{enumerate}

\item {\bf{Correspondencia}}
	Un satélite de masa $m=20$\,kg circunda a la Tierra cada $2$\,h en una
		órbita de $8060$,km de radio. 
	\begin{enumerate}
		\item Aplicando la cuantización de Bohr-Somerfeld para el momento
			angular, es decir $L=n\hbar$, calcule el número cuántico $n$ de la
			órbita del satélite.
		\item Demuestre, a partir del resultado de Bohr para la cantidad de
			movimiento angular y la ley de Newton de la gravitación universal,
			que el radio de una órbita de satélite terrestre es directamente
			proporcional al cuadrado del número cuántico, $r=k n^2$, donde $k$
			es la constante de proporcionalidad. 
		\item Con el resultado del punto anterior, determine la distancia entre
			la órbita del satélite en este problema y su siguiente órbita
			permitida. Exprese el resultado en km e interprete el resultado,
			comentando si es posible observar la separación entre dos órbitas
			adyacentes. 
	\end{enumerate}
\end{enumerate}
\end{document}
