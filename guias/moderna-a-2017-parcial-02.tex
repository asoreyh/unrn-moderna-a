\documentclass[a4paper,12pt]{article}
\usepackage[spanish]{babel}
\usepackage[utf8]{inputenc}
\usepackage[T1]{fontenc}
\usepackage{graphicx}
\usepackage[pdftex,colorlinks=true, pdfstartview=FitH, linkcolor=blue,
citecolor=blue, urlcolor=blue, pdfpagemode=UseOutlines, pdfauthor={H. Asorey},
pdftitle={Física Moderna A - Asorey - 2017}]{hyperref}
\usepackage[adobe-utopia]{mathdesign}

\hoffset -1.23cm
\textwidth 16.5cm
\voffset -2.0cm
\textheight 26.0cm

%----------------------------------------------------------------
\begin{document}
\title{
{\normalsize{Universidad Nacional de Río Negro - Profesorados de Física}}\\
Física Moderna A \\ Segundo Parcial\\}
\author{Asorey}
\date{2017}
\maketitle

\begin{enumerate}
	\item {\bf{ }}:\\
		Ciertos tipos de láseres de semiconductores (con múltiples aplicaciones
		tecnológicas) funcionan utilizando un pozo de potencial finito obtenido
		a partir de la combinación de capas de diferentes materiales, de manera
		que uno de ellos sea lo suficientemente delgado como para actuar como
		una barrera finita de potencial.  
	
	
	
	\item {\bf{Alguna línea de Lyman}}:\\
		Sea un átomo de hidrógeno con el electrón en el
		nivel fundamental que absorbe un fotón con $\lambda=102.6$\,nm y momento
		angular $L_\gamma = + 1 \hbar$.
		\begin{enumerate}
			\item Calcule la energía del fotón absorbido y la energía del
				orbital inicial y final del electrón.
			\item Diga cuáles son los números cuánticos ($n_i$, $l_i$ y $m_i$)
				del electrón en el estado inicial, escriba la función de onda
				correspondiente a este estado, $\psi(r,\theta,\phi)$, y diga a
				que distancia del núcleo será más probable encontrar al
				electrón. Finalmente, mencione si es posible o no (no hace
				falta hacer el cálculo detallado), encontrar al electrón más
				allá del punto de retorno clásico. Justifique.
			\item A partir de la conservación del momento angular, diga cuál es
				el cambio total del $L$ del sistema y, usando las reglas de
				selección, diga cuales son los posibles números cuánticos del
				electrón en el estado final ($n_f$, $l_f$ y $m_f$) teniendo en
				cuenta las reglas de selección.
			\item Utilizando notación espectroscópica, escriba los posibles
				estados finales y las correspondientes funciones de onda
				$\psi_{n,l,m}$.
			\item A partir de las reglas de selección, enumere todos los
				posibles caminos que el electrón puede seguir para volver al
				nivel fundamental.
		\end{enumerate}
\end{enumerate}
\end{document}
