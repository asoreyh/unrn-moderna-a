\documentclass[a4paper,12pt]{article}
\usepackage[spanish]{babel}
\hyphenation{co-rres-pon-dien-te}
%\usepackage[latin1]{inputenc}
\usepackage[utf8]{inputenc}
\usepackage[T1]{fontenc}
\usepackage{graphicx}
\usepackage[pdftex,colorlinks=true, pdfstartview=FitH, linkcolor=blue,
citecolor=blue, urlcolor=blue, pdfpagemode=UseOutlines, pdfauthor={H. Asorey},
pdftitle={Física Moderna A - Asorey - 2017}]{hyperref}
\usepackage[adobe-utopia]{mathdesign}

\hoffset -1.23cm
\textwidth 16.5cm
\voffset -2.0cm
\textheight 26.0cm

%----------------------------------------------------------------
\begin{document}
\title{
{\normalsize{Universidad Nacional de Río Negro - Profesorados de Física}}\\
Física Moderna A \\ Aplicaciones a casos simples\\}
\author{Asorey}
\date{2017}
\maketitle

\begin{enumerate}
	\setcounter{enumi}{40}      %% Offset en numero de problema
	%% cálculo de expectación
	%% alguno más de barrera con números
	%%
	\item {\bf{Pozos y barreras}}:\\
		Utilizando lo visto en clase, obtenga las soluciones funcionales, es
		decir, sin considerar las constantes de normalización, para las
		funciones de onda correspondientes a las siguientes situaciones
		físicas:
		\begin{enumerate}
			\item Partícula de energía $E$ en un pozo infinito de potencial de
				ancho $L$.
			\item Partícula de energía $E$ en un pozo finito de potencial de
				ancho $L$ y altura $U_0>E$.
			\item Partícula de energía $E$ en una barrera de potencial de ancho
				$\delta x$ y altura $U_0>E$.
		\end{enumerate}
		En cada caso, realice un dibujo aproximado de la situación planteada y
		de la forma (genérica) que espera para la función de onda obtenida.
	
	\item {\bf{Combinación de soluciones de un pozo}}:\\
		Hemos comprobado que una combinación lineal de dos funciones de onda
		de un sistema es también una función de onda válida para ese sistema.
		Encuentre la constante de normalización $B$ para la función de onda
		\[ \Psi = B \left ( \sin \frac{\pi x}{L} + \sin \frac{2 \pi x}{L}
		\right ), \]
		correspondiente a una combinación de los estados $n=1$ y $n=2$ de la
		partícula en una caja de ancho $L$.

	\item {\bf{Ionización del pozo de potencial}}:\\
		Un electrón está confinado en un pozo de potencial cuadrado de
		$L=1$\,nm de ancho y profundidad $U_0=10 E_\infty$, donde $E_\infty$
		corresponde a la energía del nivel fundamental de un pozo de potencial
		infinito del mismo ancho. Si el electrón está inicialmente en el nivel
		fundamental ($n=1$) y absorbe un fotón. Calcule la longitud de onda
		mínima del fotón para que el electrón pueda salir del pozo. Repita el
		cálculo anterior para un pozo de ancho $L=2$\,nm y compare el resultado
		con el anterior explicando las diferencias observadas.

	\item {\bf{Probabilidad de transmisión}}:\\
		Hemos visto que para una barrera de ancho $L$ y energía $U_0$, la
		probabilidad de que una partícula de masa $m$ y energía $E<U_0$ que
		impacta con la barrera está dada por:
		\[T=T_0 e^{-2 \beta L},\]
		donde
		\[T_0= 16 \frac{E}{U_0} \left (1 - \frac{E}{U_0} \right )
		\quad\mathrm{y}\quad 
		\beta = \frac{\sqrt{2 m (U_0 - E)}}{\hbar}.\]
		Calcule la probabilidad de que un electrón ($m_e=0.511$\,MeV/c$^2$) de
		energía $E=8$\,eV atraviese una barrera de $U_0=10$\,eV si el ancho de
		la misma es $L=1$\,nm y luego se reduce a $L=0.1$\,nm. Repita sus
		cálculos si en vez de un electrón se utilizara un muón,
		$m_\mu=105.7$\,MeV/c$^2$.

	\item {\bf{Electrón versus protón}}:\\
		Un electrón ($m_e=0.511$\,MeV$/c^2$) y un protón
		($m_p=938.3$\,MeV$/c^2$) con la misma energía se acercan a
		una barrera de potencial de altura $U_0>E$. ¿Tienen ellos la misma
		probabilidad de atravesar la barrera? En caso negativo, ¿cuál tiene
		mayor probabilidad de atravesarla?
	  
	\item {\bf{Oscilador armónico}}:\\
		Muestre que el espaciamiento de los niveles de energía de un
		oscilador armónico cuántico obedece el principio de correspondencia.
		Para ello, obtenga la relación $\Delta E_n/E_n$ entre dos niveles de
		energía adyacentes y verifique que sucede con esta relación al tomar el
		límite $n \to \infty$
	
	\item {\bf{Nivel fundamental del oscilador armónico cuántico}}:\\
		Encuentre (y dibuje) la función de probabilidad $P(x)=|\psi(x)|^2$ en
		el estado para un oscilador armónico en el estado $n=0$.
\end{enumerate}
\end{document}
%%%%
