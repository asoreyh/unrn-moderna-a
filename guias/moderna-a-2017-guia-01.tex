\documentclass[a4paper,12pt]{article}
\usepackage[spanish]{babel}
%\usepackage[latin1]{inputenc}
\usepackage[utf8]{inputenc}
\usepackage[T1]{fontenc}
\usepackage{graphicx}
\usepackage[pdftex,colorlinks=true, pdfstartview=FitH, linkcolor=blue,
citecolor=blue, urlcolor=blue, pdfpagemode=UseOutlines, pdfauthor={H. Asorey},
pdftitle={Física Moderna A - Asorey - 2017 }]{hyperref}
\usepackage[adobe-utopia]{mathdesign}

\hoffset -1.23cm
\textwidth 16.5cm
\voffset -2.0cm
\textheight 26.0cm

%----------------------------------------------------------------
\begin{document}
\title{
{\normalsize{Universidad Nacional de Río Negro - Profesorado de Física}}\\
Física Moderna A \\ La crisis de principios del Siglo XX\\}
\author{Asorey}
\date{2017}
\maketitle

\begin{enumerate}
\setcounter{enumi}{0}      %% Offset en numero de problema

\item {\bf{Ondas Electromagnéticas}}:\\
	A partir de la relación $c = \lambda f$, donde $f$ es la frecuencia,
		$\lambda$ es a longitud de onda y $c=299792458$\,m\,s$^{-1} \simeq
		3\times 10^8$\,m\,s$^{-1}$ es la velocidad de la luz, complete las columnas 1 o 2 de la siguiente tabla según corresponda:
		\begin{center}
			\begin{tabular}{|c|c|c|c|c|c|c|c|}
				\hline
				1 & 2 & 3 & 4 & 5 & 6 & 7 & 8 \\
				\hline
				$\lambda$ & $f$ (GHz) & $E_\lambda$ (J) & $E_\lambda$ (eV) & B($\lambda, 290$\,K$)$ & B($\lambda, 5700$\,K$)$ & B($\lambda, 15000$\,K$)$ & R($\lambda,15000)$\\
				\hline
				78\,m & & & & & & & \\
				\hline
				1\,m & & & & & & & \\
				\hline
				 & 0.1019 & & & & & & \\
				\hline
				730\,nm & & & & & & & \\
				\hline
				550\,nm & & & & & & & \\
				\hline
				420\,nm & & & & & & & \\
				\hline
				& $7.3 \times 10^{-4}$ & & & & & & \\
				\hline
				250\,nm & & & & & & & \\
				\hline
				& $1.58 \times 10^{3}$ & & & & & & \\
				\hline
				& $3.29 \times 10^{3}$ & & & & & & \\
				\hline
				10\,nm & & & & & & $2.625 \times 10^{-18}$ & \\
				\hline
			\end{tabular}
		\end{center}
		
\item {\bf{Energía de una onda electromagnética}}:\\
	Complete las columnas 3 y 4 de la tabla anterior, en joules y en
		electrón-voltios respectivamente. Para ello, utilice la relación para
		la energía de un fotón, $E = h f = h c / \lambda$, donde $h$ es la
		constante de Planck, $h=6.626\times 10^{-34}$\,J\,s $=4.136\times
		10^{-15}$\,eV\,s.

\item {\bf{Radiación de Cuerpo Negro}}:\\
	A partir de la expresión de Planck para la radiancia espectral de un cuerpo
		negro a temperatura $T$, $$ B(\lambda, T) = \frac{2 h c^2}{\lambda^5
		\left (e^{\frac{hc}{\lambda k T}}-1 \right )},$$ complete las columnas
		5, 6 y 7 de la tabla del ejercicio 1. Observe que las unidades de la
		radiancia espectral son las de potencia radiada (W) por unidad de
		ángulo sólido (sr) por unidad de área (m$^2$) por unidad de longitud de
		onda (m). Ayuda 1: Puede utilizar Excel para realizar estos cálculos.
		Ayuda 2: Como ejemplo, $B(10\mathrm{\ nm}, 15000\mathrm{\ K}) = 2.625
		\times 10^{-18}$\,W\,sr$^{-1}$\,m$^{-3}$.

\item {\bf{Catástrofe ultravioleta}}:\\
	Tal como hemos visto en clase, la llamada ``catástrofe ultravioleta'' se
		dio al comparar la predicción ``clásica'' para la radiancia espectral
		de un cuerpo negro a temperatura $T$, dado por la llamada ley de
		Rayleigh-Jeans, $$ R(\lambda,T) = \frac{2 c k T}{\lambda^4},$$ con los
		datos experimentales. Complete la columna 8 de la tabla del ejercicio 1
		con la mencionada ley para un cuerpo negro de $T=15000$\,K y compárela
		con la correspondiente radiancia espectral obtenida con la expresión de
		Planck para un cuerpo negro a la misma temperatura (columna 7).
		Verifique que las diferencias significativas se dan para longitudes de
		onda menores al rango visible, $\lambda \lesssim 430$\,nm (catástrofe
		del ultravioleta).

\item {\bf{Lamparita}}:\\
	Suponga que una lámpara de $100$\,W emite toda la potencia consumida en
	forma de luz monocromática de $530$\,nm. Calcule la frecuencia, la energía
	y la cantidad de movimiento ($p=E/c$) de estos fotones, y luego calcule el
	número de fotones emitidos por la lámpara cada segundo. Justifique. 

\item {\bf{Ikaros}}:\\
	Investigue el funcionamiento de la sonda espacial IKAROS y discuta la forma
	de propulsión en el marco de la teoría cuántica para los fotones.
	Interprete y responda: ¿qué es la presión de radiación?

\item {\bf{Efecto fotoeléctrico}}:\\
	La longitud de onda umbral para la emisión de fotoelectrones (electrones
	emitidos por efecto fotoeléctrico) en el cesio (Cs) es $579$\,nm. Calcule
	la función trabajo $\phi$ del Cesio en eV, y luego diga cuál es la energía
	cinética (en eV) y la velocidad máxima (en m/s) de los electrones emitidos
	cuando se ilumina una plancha de cesio con radiación electromagnética de
	$\lambda = 720\mathrm{\ nm}, 579\mathrm{\ nm}, 480\mathrm{\ nm,\ y\ }
	430$\,nm. Para ello, recuerde que la energía cinética de los electrones
	emitidos es $E_K = h f - \phi$ y que los mismos no son relativistas.  

\item {\bf{Fotómetro}}:\\
	¿Cuál debería ser la función trabajo de un material si queremos emplearlo
	para medir la intensidad de una haz de luz visible, $400$\,nm$\lesssim
	\lambda \lesssim 700$\,nm?

\item {\bf{Emisión fotoeléctrica}}:\\
	La longitud de onda umbral de los fotoelectrones en tungsteno (W) es de
	$272$\,nm. Calcula la energía cinética máxima de los electrones emitidos al
	iluminar al metal con radiación UV de $f=1.45\times 10^{15}$\,Hz. Exprese
	la respuesta en electrón-volts.

\item {\bf{Efecto Compton}}:\\
	El efecto Compton corresponde a la dispersión inelástica de fotones con
	partículas cargadas libres. Según esto, el fotón incidente de longitud de
	onda $\lambda$ es dispersado a un ángulo $\theta$ respecto a su dirección
	de propagación original y emerge con una longitud de onda $\lambda' >
	\lambda$, según la siguiente relación: $$ \lambda' - \lambda =
	\left(\frac{h}{mc}\right) \left ( 1 - \cos(\theta) \right ).$$ Al primer
	factor se lo conoce como longitud de onda Compton de la partícula en
	cuestión. Verifique que esta expresión tiene unidades de longitud y luego
	calcule la longitud de onda Compton para el electrón, $\lambda_e = h / (m_e
	c)$, para el muón, $\lambda_\mu = h / (m_\mu c)$ y para el protón
	$\lambda_p = h / (m_p c)$. Exprese el resultado en picómetros (1\,pm $=
	10^{-12}$\,m).

\item {\bf{Longitud de onda Compton}}:\\
	A partir de la expresión para el cambio de longitud de onda del fotón,
	obtenga la siguiente relación para las energías del fotón incidente y
	emitido: $$\frac{1}{E_{\lambda'}} - \frac{1}{E_\lambda} = \frac{1}{m c^2}
	\left(1 - \cos(\theta) \right).$$ La diferencia de energía es transferida a
	la partícula de masa $m$, que adquiere una energía cinética $E_k =
	E_\lambda - E_{\lambda'}$. 

\item {\bf{Retrodispersión Compton}}:
	A partir de la expresión para el cambio de longitud de onda, $$ \Delta
	\lambda \equiv \lambda' - \lambda = \lambda_e \left ( 1 - \cos(\theta)
	\right ),$$ calcule el $\Delta \lambda$ para un electrón cuando $\theta =
	0, \pi/4, \pi/2, 3\pi/4, \pi$. Luego, calcule en cada caso la energía del
	fotón dispersado y la energía cinética transferida al electrón si el fotón
	incidente es un rayo X con $E_\lambda = 1$\,keV. 

\item {\bf{Función angular}}:\\
	Utilizando Libreoffice calc (o Excel), realice un gráfico de la función
	$\Delta \lambda = \lambda_e (1 - \cos(\theta))$ como función de $\theta$
	para un electrón.

\item {\bf{Dispersión Compton }}:\\
	Un haz de rayos X con energía $E=25$\,keV es dispersado por Compton. ¿Cuál
	es la máxima longitud de onda $\lambda'$ que podrá encontrar en los rayos X
	dispersados? ¿En qué ángulo? ¿Espera observar rayos X de $25$\,keV entre
	los dispersados? En caso afirmativo, ¿en qué ángulo? ¿Cuáles son las
	energías mínimas y máximas que espera observar entre los electrones
	dispersados? 
%%% 14 ejercicios
\end{enumerate}
\end{document}
%%%%
