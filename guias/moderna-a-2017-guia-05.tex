\documentclass[a4paper,12pt]{article}
\usepackage[spanish]{babel}
\hyphenation{co-rres-pon-dien-te}
%\usepackage[latin1]{inputenc}
\usepackage[utf8]{inputenc}
\usepackage[T1]{fontenc}
\usepackage{graphicx}
\usepackage[pdftex,colorlinks=true, pdfstartview=FitH, linkcolor=blue,
citecolor=blue, urlcolor=blue, pdfpagemode=UseOutlines, pdfauthor={H. Asorey},
pdftitle={Física Moderna A - Asorey - 2017}]{hyperref}
\usepackage[adobe-utopia]{mathdesign}

\hoffset -1.23cm
\textwidth 16.5cm
\voffset -2.0cm
\textheight 26.0cm

%----------------------------------------------------------------
\begin{document}
\title{
{\normalsize{Universidad Nacional de Río Negro - Profesorados de Física}}\\
Física Moderna A \\ El átomo de hidrógeno\\}
\author{Asorey}
\date{2017}
\maketitle

\begin{enumerate}
	\setcounter{enumi}{47}      %% Offset en numero de problema
	\item {\bf{Soluciones}}:\\
		Muestre que $T_2^0(\theta) = \frac{\sqrt{10}}{4} (3 \cos^2 \theta - 1)$
		es solución de la ecuación cenital, y escriba la solución a la ecuación
		radial correspondiente (sáquela de la tabla de soluciones encontrando
		el valor de $n$ correspondiente). Luego, verifique que $R_1^0(\theta) =
		\frac{2}{a_0^{3/2}} e^{-r/a_0}$ es una solución de la ecuación radial.

	\item {\bf{Cuantización de $L$}}:\\
		Compare el valor del momento angular para un electrón en el nivel
		fundamental ($n=1$) y en el nivel ($n=2, l=1$) predichos por la teoría
		de Bohr, comparado con el resultado cuántico correcto, $L=\sqrt{l(l-1)}
		\hbar$. Luego, verifique para el caso $l=n-1$, a partir de que valor de
		$n$ ambas predicciones difieren en menos de un $5\%$. 

	\item 

\end{enumerate}
\end{document}
%%%%
