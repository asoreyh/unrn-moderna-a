\documentclass[a4paper,12pt]{article}
\usepackage[spanish]{babel}
\hyphenation{co-rres-pon-dien-te}
%\usepackage[latin1]{inputenc}
\usepackage[utf8]{inputenc}
\usepackage[T1]{fontenc}
\usepackage{graphicx}
\usepackage[pdftex,colorlinks=true, pdfstartview=FitH, linkcolor=blue,
citecolor=blue, urlcolor=blue, pdfpagemode=UseOutlines, pdfauthor={H. Asorey},
pdftitle={Física Moderna A - Asorey - 2017}]{hyperref}
\usepackage[adobe-utopia]{mathdesign}

\hoffset -1.23cm
\textwidth 16.5cm
\voffset -2.0cm
\textheight 26.0cm

%----------------------------------------------------------------
\begin{document}
\title{
{\normalsize{Universidad Nacional de Río Negro - Profesorados de Física}}\\
Física Moderna A \\ El átomo de hidrógeno\\}
\author{Asorey}
\date{2017}
\maketitle

\begin{enumerate}
	\setcounter{enumi}{47}      %% Offset en numero de problema
	
	\item {\bf{Cuantización de $L$}}:\\
		Compare el valor del momento angular para un electrón en el nivel
		fundamental ($n=1$) y en el nivel ($n=2, l=1$) predichos por la teoría
		de Bohr, comparado con el resultado cuántico correcto, $L=\sqrt{l(l-1)}
		\hbar$. Luego, verifique para el caso $l=n-1$, a partir de que valor de
		$n$ ambas predicciones difieren en menos de un $5\%$. 

	\item {\bf{Cuantización de la energía}}:\\
		Verifique los cálculos realizados en clase y muestre que la energía de
		un estado con número cuántico principal $n$ es $E_n=E_1 / n^2$, donde
		$$E_1= -\frac{1}{(4\pi\epsilon_0)^2} \frac{m e^4}{2\hbar^2},$$ y que
		$E_1=-13.6$\,eV. Luego, siga los lineamientos de la clase para
		encontrar que el valor de $a_0$ corresponde con el radio de Bohr.

	\item {\bf{Números cuánticos}}:\\ 
		Para un electrón en el nivel $n=5$, escriba todos los posibles
		valores de los números cuánticos $l$ y $m$, el nombre del orbital, y el
		número total de estados diferentes que el nivel $n=5$ puede tener
		(recuerde que la multiplicidad es $(2l+1)$ para cada valor de $l$).
		Luego, para el caso $l=4$, calcule los posibles valores de $L$ y $L_z$,
		y los posibles ángulos que forma el vector $\vec L$ con el eje $z$.

	\item {\bf{Soluciones}}:\\
		Muestre que $T_2^0(\theta) = \frac{\sqrt{10}}{4} (3 \cos^2 \theta - 1)$
		es solución de la ecuación cenital, y escriba la solución a la ecuación
		radial correspondiente (sáquela de la tabla de soluciones encontrando
		el valor de $n$ correspondiente). Luego, verifique que $R_1^0(\theta) =
		\frac{2}{a_0^{3/2}} e^{-r/a_0}$ es una solución de la ecuación radial.

	\item {\bf{Solución radial}}:\\
		Utilizando la regla de normalización de la solución radial y la
		expresión para calcular los polinomios de Laguerre vista en clase,
		verifique que la solución correspondiente al nivel $R_3^1$ es $R_3
		^1=\frac{4}{81 \sqrt{6} a_0^{3/2}} \left (\frac{r}{a_0} \right ) \left (6 - \frac{r}{a_0} \right ) \exp\left (-\frac{r}{3 a_0}\right )$.
		Luego, a partir de las expresiones para los armónicos esféricos,
		encuentre la función de onda del estado
		$\psi_{3,1,+1}(r,\theta,\varphi)$. Compárela con la expresión vista en
		clase (Tabla 11/29 U05C03).

	\item {\bf{Probabilidades}}:\\
		Para un electrón en un átomo de hidrógeno, la probabilidad de
		encontrarlo en un cascarón esférico de radio interior $r$ y radio
		exterior $r+dr$ (espesor $dr$) es $P(r)=r^2 |R_n^l|^2 dr$. Encuentre el
		valor de $r$ para el cual la probabilidad de encontrar al electrón en
		el estado $1s$ es máxima, y compare esta valor con la predicción del
		modelo de Bohr para el nivel fundamental.

	\item {\bf{Punto de viraje clásico}}:\\
		Calcule para que valor de $r$, la energía total del nivel fundamental
		del átomo de hidrógeno ($n=1$) es igual al valor de la energía
		potencial $U(r) = e^2/(4 \pi \epsilon_0 r)$. A este valor, $r_c$ se lo
		conoce como punto de viraje clásico, ya que en el análisis clásico el
		electrón siempre estará confinado a $r\leq r_c$, cosa que no ocurre
		para el análisis cuántico. Luego, calcule el valor $P(r_2)=r_c^2
		|R_1^0(r_c)|^2$ de encontrar al electrón en un entorno de $r_c$,
		verificando que es mayor que cero, y que por lo tanto hay una
		probabilidad no nula de encontrar al electrón en $r>r_c$.
	
	\item {\bf{Cercanos y lejanos}}:\\
		Para un electrón en el nivel $1s$, calcule el valor de
		$\psi(r,\theta,\varphi)$ para $r=a_0$ y $r=a_0/2$. Luego, calcule la
		probabilidad de que un electrón en ese estado se encuentre entre
		$a_0/2$ y $a_0$, entre $a_0$ y $2 a_0$, y luego entre $5 a_0$ y $6
		a_0$.

	\item {\bf{Reglas de selección}}:\\
		A partir de las reglas de selección para las transiciones entre dos
		estados del átomo de hidrógeno, $\psi_{n' l' m_l'} \to \psi_{n l m_l}$:
		$\Delta l \equiv l'-l = \pm 1$ y $\Delta m_l \equiv m_l' - m_l = 0, \pm
		1$, encuentre cuales transiciones son permitidas y cuales son
		prohibidas para una transición $4f \to 3d$, para una transición $3d
		\to 2p$ y finalmente para la transición $4f \to 2p$. Verifique si es
		más conveniente para el electrón (es decir, si hay más transiciones
		permitidas) para la transición directa o para aquellas que pasan por el
		nivel $3d$. 
\end{enumerate}
\end{document}
%%%%
